\section{The Later Modality}
\label{sec:introducing-later}

\newcommand{\lobrule}[1][]
{\rulegenhref[#1]{L{\"o}b}{Loeb}
  {Q \land \later\prop \proves \prop}
  {Q \proves \prop}}
\newcommand{\latermonorule}[1][]
{\rulegenhref[#1]{later-mono}{Later-Mono}
 {Q \proves \prop}
 {\later Q \proves \later\prop}}
\newcommand{\laterweakrule}[1][]
{\rulegenhref[#1]{later-weak}{Later-weak}
{Q \proves \prop}
{Q \proves \later{\prop}}}
\newcommand{\laterexistsrule}[1][]
{\rulegenhref[#1]{$\later$-$\exists$}{later-exists}
  {\text{$\type$ is inhabited} \and Q \proves \later{\Exists x:\type.\prop}}
  {Q \proves \Exists x:\type. \later\prop}}
\newcommand{\existslaterrule}[1][]
{\rulegenhref[#1]{$\exists$-$\later$}{exists-later}
  {Q \proves \Exists x. \later P}
  {Q \proves \later{\Exists x.\prop}}}
\newcommand{\laterconjrule}[1][]
{\rulegenhrefb[#1]{later-$\wedge$}{Later-conj}
  {R \proves \later (P \land Q)}
  {R \proves \later P \land \later Q}}
\newcommand{\laterdisjrule}[1][]
{\rulegenhrefb[#1]{later-$\vee$}{Later-disj}
  {R \proves \later (P \lor Q)}
  {R \proves \later P \lor \later Q}}
\newcommand{\laterforallrule}[1][]
{\rulegenhrefb[#1]{later-$\forall$}{Later-all}
  {Q \proves \later{\All x.P}}
  {Q \proves \All x. \later P}}
\newcommand{\laterseprule}[1][]
{\rulegenhrefb[#1]{later-$\ast$}{Later-sep}
  {R \proves \later P \ast \later Q}
  {R \proves \later (P \ast Q)}}

\newcommand{\htbetalater}[1][]{\htbetagen[-later#1]{\later}{ }}
\newcommand{\htloadlaterrule}[1][]{\htloadgen[#1]{\later}}
\newcommand{\htstorelaterrule}[1][]{\htstoregen[#1]{\later}}
\newcommand{\htreclaterrule}[1][]{\htrecgen[#1]{\later}}
\newcommand{\htmatchlaterrule}[1][]{\htmatchgen[#1]{\later}}
\newcommand{\htletlaterrule}[1][]{\htletgen[#1]{\later}}
\newcommand{\htletdetlaterrule}[1][]{\htletdetgen[#1]{\later}}
\newcommand{\htseqlaterrule}[1][]{\htseqgen[#1]{\later}}
\newcommand{\htiflaterrule}[1][]{\htifgen[#1]{\later}}

The \emph{later modality} is an essential feature of Iris.  It will be
used extensively in connection with invariants which we introduce in
Section~\ref{sec:invar-ghost-state}.  However it can also be used for
other things, chiefly for specifying and reasoning about higher-order
programs using higher-order store.  We show a more involved example of
this in
Section~\ref{sec:case-study:sequential-invariants}.  In
this section we introduce the later modality and its associated
powerful L\"ob induction principle by a small, and rather artificial,
example which has the benefit that it is relatively simple and only
involves one new concept.

Recall that in untyped lambda calculus one can define a fixed-point
combinator which can be used to express recursive functions.  In our
\proglang\ language we have a primitive fixed point combinator $\Rec{f} x = e$
but, for the purposes of introducing the later modality, we
will show how to define a fixed-point combinator and prove a suitable 
specification for it. 

To this end, we assume the following specification for $\lambda$ terms.
\begin{mathpar}
  \infer
  {S \proves \hoare{P}{e\left[v/x\right]}{u.Q}}
  {S \proves \hoare{P}{(\lambda x . e) v}{u.Q}}
\end{mathpar}
%
Given a value $F$, the call-by-value Turing fixed-point combinator
$\Theta_F$ is the following term
\begin{align*}
  \Omega_F &= \lambda r . F (\lambda x . r r x)\\
  \Theta_F &= \Omega_F \Omega_F
\end{align*}
One can easily derive (exercise!), for any values $F$ and $v$,
\begin{align*}
  \Theta_F v \stepsto F (\lambda x . \Theta_F x) v
\end{align*}
and thus, if $F = \lambda f x . e$ then one should think of 
$\Theta_F$ as $\Rec{f} x = e$.

Now we wish to derive a useful proof rule for $\Theta_F$, analogous to the simplified version of the \ruleref{Ht-Rec} rule displayed in Figure~\ref{fig:hoare-triple-rules-sequential} on page \pageref{fig:hoare-triple-rules-sequential}.
\footnote{The rule is simplified by the removal of the ``logical'' variables $y$ in the rule \ruleref{Ht-Rec}. These are not essential for understanding this example and would only complicate the reasoning.}
\begin{mathpar}
  \inferH{Ht-Turing-fp}
  {\Gamma \mid S \land \All v. \hoare{P}{\Theta_F v}{u.Q} \proves \All v. \hoare{P}{F (\lambda x . \Theta_F x) v}{u.Q}}
  {\Gamma \mid S \proves \All v. \hoare{P}{\Theta_F v}{u.Q}}
\end{mathpar}
But at present there is nothing in the logic which would allow us to do so.
In contrast to the specifications of the linked list methods above, there is no tangible structure, such as the mathematical sequence representing a list, which could be used as a crutch to prove the specification by induction, for example.

We are thus led to extend the logic with a new construct, the $\later$ (pronounced ``later'') modality.
The main punch of the later modality is the L\"ob rule:
\begin{mathpar}
  \lobrule[-inline]
\end{mathpar}
which is similar to a coinduction principle.
It states that (in any context $Q$), if from $\later P$ we can derive
$P$, then we can also derive $P$ without any assumptions.
When using this rule we will often say we ``proceed by L\"{o}b induction'', or that we are going to use ``L\"{o}b induction''.

Note that the $\later$ modality is necessary for the rule to be sound:
if we admit the rule
\begin{align}
  \label{eq:inconsistent-rule}
  \infer
  {Q \land \prop \proves \prop}
  {Q \proves \prop}
\end{align}
to the logic, then the logic would become inconsistent,
in the sense that we could then prove $\TRUE \proves \FALSE$.
\begin{exercise}
  Assuming \eqref{eq:inconsistent-rule} derive $\TRUE \proves \FALSE$.
\end{exercise}


Intuitively, the $\later$ modality in the premise $\later P$ in the
\ruleref{Loeb} rule
ensures that when proving $P$, we need to ``do some work'' before we 
can use $P$ as an assumption.

The rest of the rules for the $\later$ modality, listed in
Figure~\ref{fig:laws-for-later}, essentially ensure that we can
get the later modality at the correct place in order to use the L\"ob
rule.

\begin{figure}[htbp]
  \centering
  \begin{mathpar}
    \latermonorule
    \and
    \laterweakrule
    \and
    \lobrule
    \and
    \laterexistsrule
    \and
    \existslaterrule
    \and
    \laterconjrule
    \and
    \laterdisjrule
    \and
    \laterforallrule
    \and
    \laterseprule
  \end{mathpar}
  \caption{Laws for the later modality.
    A type $\type$ is \emph{inhabited} if $ \proves \wtt{\term}{\type}$ is derivable for some $\term$.
  }
  \label{fig:laws-for-later}
\end{figure}

\paragraph{Strengthening the Hoare rules}
In order for the later modality to be useful for proving specifications (Hoare triples) we will need to connect it in some way to the steps a program can take.
Let us see how this comes up by trying to show \ruleref{Ht-Turing-fp}.

We proceed by L\"ob induction so we assume $\later \All v.\hoare{P}{\Theta_F v}{u.Q}$ and we are to show
\begin{align*}
  \All v.\hoare{P}{\Theta_F v}{u.Q}.
\end{align*}
Let $v$ be a value.
By using the \ruleref{Ht-bind} and \ruleref{Ht-beta} it suffices to show $\hoare{P}{F (\lambda x .
  \Theta_F x) v}{u.Q}$ and thus by the assumption of the rule we are proving it suffices to show
\begin{align*}
  \All v.\hoare{P}{\Theta_F v}{u.Q}.
\end{align*}
However we only have the L\"ob induction hypothesis
\begin{align*}
  \later\All v.\hoare{P}{\Theta_F v}{u.Q}
\end{align*}
from which we cannot get what is needed.
The issue is that we have not connected the later modality to 
the programming language constructs in any way.
One way to do that is to have a stronger \ruleref{Ht-beta} rule,
which only assumes $\later P$ in the precondition of the triple in
the conclusion of the rule:
\begin{mathpar}
  \htbetalater
\end{mathpar}
The intuitive explanation for why this rule is sensible is that the
term $(\lambda x.e)v$ takes one more step to evaluate than
$e\left[v/x\right]$; hence any resources accessed by the body $e$
will only be needed one step \emph{later}.

\begin{exercise}
  Convince yourself that the old beta rule is derivable from the new one using the rules presented thus far.
\end{exercise}


The final piece we need is that if $P$ is a persistent proposition
(\eg{} a Hoare triple or equality) then $\later P$ is also a
persistent proposition, which implies that we can move it in and out of
the preconditions (\emph{c.f.} \ruleref{Ht-Ht}).

Let us prove the rule \ruleref{Ht-Turing-fp}.
We proceed by L\"ob induction so we assume
\begin{align*}
  \later \All v.\hoare{P}{\Theta_F v}{u.Q}
\end{align*}
and we are to show
\begin{align*}
  \All v.\hoare{P}{\Theta_F v}{u.Q}.
\end{align*}
Let $v$ be a value.
By using \ruleref{Later-weak} and the rule of consequence \ruleref{Ht-csq} it suffices to show $\hoare{\later P}{\Theta_F v}{u.Q}$.
Since Hoare triples are persistent propositions this is equivalent to showing
\begin{align*}
  \hoare{\later (\All v.\hoare{P}{\Theta_F v}{u.Q} \land P)}{\Theta_F v}{u.Q}
\end{align*}
By the bind rule and the stronger rule \ruleref{Ht-beta} introduced above it thus suffices to show
\begin{align*}
  \hoare{\All v.\hoare{P}{\Theta_F v}{u.Q} \land P}{F (\lambda x.\Theta_F x) v}{u.Q}
\end{align*}
which again is equivalent (rule \ruleref{Ht-Ht}) to showing
$\hoare{P}{F (\lambda x.\Theta_F x) v}{u.Q}$ assuming
$\All v.\hoare{P}{\Theta_F v}{u.Q}$.  But this is exactly the premise
of the rule \ruleref{Ht-Turing-fp},  and thus the proof is concluded.

\subsection{Stronger rules for Hoare triples}
\label{sec:stronger-rules-for-hoare-triples}

With the introduction of the later modality, we can strengthen the
rules for Hoare triples, so that they allow the removal of the later
modality in preconditions in those cases where the term we are specifying is
not a value.  From now on, we will use these stronger rules.  We only list
the rules which differ.

\begin{mathparpagebreakable}
  \htloadlaterrule
  \and
  \htstorelaterrule
  \and
  \htreclaterrule
  \and
  \htmatchlaterrule
\end{mathparpagebreakable}

\begin{exercise}
  Derive the rules in Section~\ref{sec:basic-separation-logic} apart from \ruleref{Ht-Rec}, for which see Exercise~\ref{exercise:derived-rule-recursive-functions} below, from the rules just listed.
\end{exercise}


\begin{exercise}
  \label{exercise:derived-rule-let-expression-sequencing-stronger}
  This is a variant of Exercise~\ref{exercise:derived-rule-let-expression-sequencing} above deriving stronger rules for Hoare triples which allow us to remove more later modalities.

  Show the following rules
  \begin{mathpar}
    \htletlaterrule
    \and
    \htletdetlaterrule
    \and
    \htseqlaterrule
    \and
    \htiflaterrule
  \end{mathpar}
\end{exercise}

In the rest of the document we use these stronger variants of the
rules.
(For instance, when in future sections we refer to the sequencing rule, we refer to 
the one above involving a $\later$.)


\subsection{Recursively defined predicates}
\label{sec:recursively-defined-pred}

With the addition of the later modality we can extend the logic with \emph{recursively defined predicates}.
The terms of the logic are thus extended with
\begin{align*}
    \term \bnfdef \cdots \mid \MU \var:\type. \term
\end{align*}
with the side-condition that the recursive occurrences must be \emph{guarded}: in $\MU \var.\term$, the variable $\var$ can only appear under the later $\later$ modality.

The typing rule for the new term is as expected
\begin{align*}
  \infer{
  \vctx, \var:\type \proves \wtt{\term}{\type} \and
  \text{$\var$ is guarded in $\term$}
  }{
  \vctx \proves \wtt{\MU \var:\type. \term}{\type}
  }
\end{align*}
where $x$ is \emph{guarded in} $t$ if all of its occurrences in $t$
are under the $\later$ modality.  For example, $P$ is guarded in the following
terms (where $Q$ is a closed proposition)
\begin{mathpar}
  \later(P \land Q) \and  (\later P) \implies Q \and Q    
\end{mathpar}
but is \emph{not} guarded in any of the following terms
\begin{mathpar}
  P \land \later Q \and  P \implies Q \and P \lor Q.
\end{mathpar}
%
We have a new rule for using guarded recursively defined
predicates, which expresses the fixed-point property:
\begin{mathpar}
  \inferH{Mu-fixed}
  { }
  {Q \proves \MU \var:\type.\term =_\type t\left[\MU\var:\type.\term/\var\right]}
\end{mathpar}
This rule is typically used in conjunction with L\"ob induction.

We shall see examples of the use of guarded recursively defined
predicates later on.




%%% Local Variables:
%%% mode: latex
%%% TeX-master: "../main.tex"
%%% End:
